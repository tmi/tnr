\chapter{Lecture 6}
\centerline{\tt the Lavrentieff's theorem is kind of incomplete...}

% notes in this lecture are damn cryptic, be aware!

{\bf Theorem. } Let $(M,\rho),\ (N,\sigma)$ be metric spaces, $N$ is complete,
$A\subseteq M$ be dense in $M$. If $f\,:A\to N$ is continuous, then there is a 
$G_\delta$ set $B\supseteq A$ and $F\,:B\to N$, continuous function $F\supseteq 
f$.
\smallskip
{\it Proof:} set $O_n = \bigcup \{G,\ G\subseteq M$ is open, $\forall x,y\in 
G\cap A\ \sigma(f(x),f(y)\leq 1/n\}$. We see that every $O_n$ is open (as a 
union of open sets $G$), but it also holds that $A\subseteq O_n$ -- we pick 
$x\in A$, then $\exists \delta > 0\ \rho(x,y) < \delta \imply \sigma (f(x),
f(y) \leq 1/(2n)$ (since $f$ is continuous), therefore for any $y,y' \in 
B_{\delta}(x)$ is $\sigma(f(y),f(y')) \leq \sigma(f(y),f(x)) + \sigma(f(y')
,f(x)) \leq 1/n$. We set $B=\bigcap O_n$, then $B$ is $G_\delta$ set containing 
$A$. Now, we proceed with construction of $F$:
{\parindent0.5in\parskip6pt
	\item{$\bullet$} for $z\in A$ set $F(z) = f(z)$,
	\item{$\bullet$} for $z\in B\bs A$ pick a sequence $(x_n^z,\ n\in \N,\ 
		x_n^z\in A)$ such that this sequence converges to $z$ (we can do this 
		since $A$ is dense), and set $F(z) = \lim_{n\to \infty} f(x_n^z)$. To 
		be sure that $F(z)$ is well-defined, we need to check that $f(x_n^z)$
		is Cauchy: as $f$ is continous, there is $\delta$ such that $\rho(x,
		y)\leq \delta\imply \sigma(f(x),f(y))$. $x_n^z$ itself is convergent 
		and therefore Cauchy, so we can pick $n_0$ such that $\forall m,n>n_0$
		$\rho(x_m^z,x_n^z) < \delta$ and $f(x_m^z,x_n^z) < \varepsilon$.

}
Clearly is $F\supseteq f$, so we just have to check that $F$ is continuous.
We have an $\varepsilon$ and seek the right $\delta$. Let us pick arbitrary 
$z = (x_n^z),\ y=(x_m^y)$ (if either $y$ or $z$ are already in $A$, we set the 
sequence identical), and bound distance of their images: $\sigma(f(z),f(y)) 
\leq \sigma(f(z),f(x_n^z)) +\sigma(f(x_n^z),f(x_m^y)) +\sigma(f(x_m^y),f(y))$.
As $f(x_n^z)$ converges to $f(z)$, we can pick $n_0$ such that $\sigma(f(z),
f(x_{n_0..}^z))$ is $<\varepsilon/3$, same for $\sigma(f(x_m^y),f(y))$. If 
$\rho(z,y)$ is lesser than some $\delta$, then we can pick $n_1$ and $m_1$ 
such that $\rho(x_{n_1..}^z,x_{m_1..}^y) < 3\delta$. Since $f$ is continuous, 
we can pick $\delta_1$ such that $\rho(a,b) < \delta_1 \imply \sigma(f(a),f(b))
< \varepsilon/3$, and then we set $\delta = \delta_1/3$; and our bounding is 
done.
\qed

\medskip
% proof taken from Engelking
{\bf Theorem (Lavrentieff).} (?) Let $M,N$ be complete metric spaces, 
$A\subseteq M$ and $B\subseteq N$ dense sets, $f\,:A\to B$ homeomorphism, then 
there are $G_\delta$ sets $C\supseteq A$ and $D\supseteq B$ and homeomorphism 
$F\,: C\to D,\ f\subseteq F$.
\smallskip
{\it Proof:} Let $g$ be the inverse of $f$, from previous theorem we know that 
there are extensions $F_0\,:C_0\to Y$ and $G_0\,:D_0\to X$ with $C_0\supseteq 
A$ and $D_0 \supseteq B$ $G_\delta$-sets. Set $C = C_0\cap G_0^{-1}[D_0]$ and 
$D = D_0 \cap F_0^{-1}[C_0]$, both are still $G_\delta$ sets; clearly with 
$A\subseteq C$ and $B\subseteq D$. Now, we show that $G_0(F_0(x)) = x$ for every 
$x\in C$ (which implies $F_0[C]\subseteq D$, since $F_0[C] \subseteq G_0^{-1} 
[C] \subseteq G_0^{-1}[C_0] = D$). {\tt missing shit}. Symmetrically, we can 
assume that $F_0(G_0(x)) = x\ \forall x\in D$ and $G_0[D]\subseteq C$; therefore 
$G_0[D] = C$ and $F_0[C]=D$. If we set $F=F_0\uparrow C$ and $G=G_0\uparrow D$, 
both remain continuous and are mutually inverse, therefore $F$ is homeomorphism.
\qed
\medskip


% subcover ? sub-cover
\tpc{Compactness} We say that metric space $(M,\rho)$ is {\it compact}, if for 
each cover $\cal C$ of $M$ by open balls there is a finite sub-cover $\cal 
C'\subseteq C$ of $M$. Following statements are equivalent:
{\parindent0.5in\parskip6pt
	\item{a)} $M$ is compact.
	\item{b)} Every family $\cal F$ of closed sets, such that any finite 
		subset of $\cal F$ has nonempty intersection, has nonempy intersection.
% limit point? hromadny bod;
	\item{c)} Every infinite subset $X$ of $M$ has a limit point (a point 
		$x$ such that any ball around $x$ contains $y\in X,\ y\neq x$).
	\item{d)} Every sequence has a convergent subsequence.
	\item{e)} Every continuous function $f\,:M\to \R$ is bounded.
	\item{f)} Every continuous function $f\,:M\to \R$ attains its supremum and 
		infimum.
	\item{g)} For every countable cover of $M$ there is a finite subcover.

}

\smallskip
e $\imply$ f: let $f$ be continuous not attaining its sup -- $t=\sup f(x) 
\imply \neg \exists x_0,\ f(x_0) = t$. Define $g\,:(-\infty,t)\to \R,\ 
g(y)=1/(t-y)$, $g$ is continous, $g\circ f$ is continous but unbounded.
\smallskip
f $\imply$ e: let $f$ be continuous attaining its supremum and infimum. 
But that means that $f[M] \subseteq [f(\inf),f(\sup)]$ and $f$ is bounded.
\smallskip
a $\imply$ b: let $\bigcap {\cal F} = \es$, then $\{M\bs F,\ F\in {\cal F}\}$ 
is open cover of $M$. This implies there is a finite open sub-cover -- 
$M = M \bs \bigcap F_{1..n}$, meaning that $\bigcap F_{1..n} = \es$.
\smallskip
b $\imply$ a: let $\cal U$ be a set of open balls without any finite sub-cover
of $M$. Then, ${\cal F} = \{M\bs U,\ U\in{\cal U}\}$ has the 
finite-intersection property and therefore $\bigcap \cal F \neq \es$, meaning 
that $\cal U$ did not cover $U$.
\smallskip
a $\imply$ c: let $X$ be a subset of $M$ without limit point. Then for all 
$x\in M$ there is an open set $G_x\ni x$ with finite intersection with $M$.
But such $G_x$ form cover of $M$, therefore there is a finite sub-cover 
$H_x$, so that $\bigcup H_x = M \supseteq X$ and each $X\cap H_x$ is finite, 
therefore $X$ is finite.
