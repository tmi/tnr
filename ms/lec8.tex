\chapter{Lecture 8}

\proclaim Theorem. For $(M,\rho)$ compact metric space, $(N,\sigma)$ metric
space and $f\,:M \to N$ continuous is $f[M]$ closed subset of $N$.

{\it Proof:} We already know that continuous image of compact space is a compact
space, and also that compact space is totally bounded and complete. Therefore
if there was a point in $x\in \overline{f[X]} \bs f[X]$, we could construct a 
Cauchy sequence $(x_n|x_n \in f[M], \sigma(x_n,x)<1/n)$ with limit $x$, which 
wouldn't have a limit in $f[X]$ -- a contradiction.
\qed

\proclaim Corollary. For compact metric space $(M,\rho)$, metric space 
$(N,\sigma)$ and $f$ continuous bijection, we have that $f$ is homeomorphism.

{\it Proof:} If $X\subseteq N$ is closed, then $(f^{-1})^{-1}[X] = f[X]$ is
closed due to previous theorem, therefore $f^{-1}$ is continuous.
\qed

\tpc{Cantor Set} In $\R$, we say that for $$C_0 = [0,1], C_i = C_{i-1} \bs 
\bigcup_{j=0}^{3^{i-1}} (j\cdot 3^{-i+1} + 3^{-i}, j\cdot 3^{-i+1} + 2\cdot 
3^{-i})$$
is ${\cal C}=\bigcup C_i$ the {\it Cantor set}. This set has many funny 
properties, for example it consist only of isolated points and therefore has 
no interior (no open ball is subset of it) and is closed (and compact, as we 
are in $\R$).

Alternatively, we can say that $\cal C$ consists of points in $[0,1]$ that have 
no 1 in their ternary expansion. This argument implies that cardinality of
$\cal C$ is continuum. This ternary expansion gives us more: if we consider 
the space $(\{0,1\}^{\omega_0},\iota)$ -- where $\iota((x_n),(y_n)) = \sum_i 
[x_i \neq y_i]\cdot 2^{-i}$, and $f\,:{\cal C} \to \{0,1\}^{\omega_0}$, 
$f(x) = x/2$; we get that $\cal C$ is bijection and continuous -- because
%a $2^{-j}$-open ball around some $x\in \{0,1\}^{\omega_0}$ consists of
%sequences that have first $j$ members same as $x$, therefore every open set has
%this property and its image in $f^{-1}$ is some set of sequences in $\cal C$
%that have the same prefix (altogether), which is too an open set. Therefore, $f$ 
%is homeomorphism.

We can construct a similiar set, by removing subintervals at $(1/5,2/5)$ and
$(3/5,4/5)$ instead of $(1/3,2/3)$. Later, we shall see that this set is 
homeomorphic to $\cal C$. With more dimensions there is more fun. If we take 
unit square and remove the middle subrectangle (height 1, width $1/3$) in 
a simillar manner; we get a 2-dimensional $\cal C$. Such a set has still big
cardinality and no interior, and has area 0 (total sum of area of removed
squares is 1, the area of unit square). On the other hand, if the removed
subrectangle has width $0 < \alpha < 1/3$, the area is nonzero and resulting
set still doesn't have any interior.

\proclaim Theorem. Every compact space is continous image of $\cal C$.

{\it Remark:} One shall note that every compact space has cardinality $\leq 
\cal C$, since {\tt to be finished.}

{\it Proof:} {\tt any volunteers?}
\qed

{\tt And here again, something is missing.}
