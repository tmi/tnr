\chapter{Lecture 3}

% uplne omezene prostory?
% epsilon sit
\tpc{Totally bounded spaces} Let $(M,\rho)$ be a metric space and 
$\varepsilon>0$, then we call $N\subseteq M$ an $\varepsilon$-net 
if $\forall x,y \in M,\ x\neq y \imply \rho(x,y) \geq \varepsilon$. 
$(M,\rho)$ is {\it totally bounded}, if every $\varepsilon$-net 
is finite. For example, $(0,1)$ is totally bounded for $\rho=|x-y|$,
but $\R$ in such metrics is not (therefore, totally bounded space 
is a topological term).
\smallskip
If $(M,\rho)$ is totally bounded, then any $N\subseteq M$ is totally 
bounded too (clearly any $\varepsilon$-net in $N$ is still a net in 
$M$). Also, if $(N,\rho_{N}),\ N\subseteq M$ is totally bounded,
then $(\overline N,\rho_{\overline N})$ is totally bounded. To see 
this, we take a $\varepsilon$-net $X$ in $\overline N$. For every 
$x$ in $\overline N \bs N$ there is $x'$ in $N$ such that $\rho(x,x')
< \varepsilon/3$ and call this substituted set $X'$. Such $x'$ is 
unique for every $x$, since $X$ was an $\varepsilon$-net. Even 
more, $\rho(x',y)\leq \varepsilon/3$ for any $y\in X'$ due to triangle 
inequality. Therefore, we can assign to any $varepsilon$-net in 
$\overline N$ a $\varepsilon$-net in $N$ of the same cardinality.
\smallskip
Let $(M_i,\rho_i)_{i\in \N}$ be a family of non-empty metric spaces,
such that $\rho_i \leq 1$ for every $i$. Then $\prod_{i\in \N} 
(M_i,\rho_i) = (M,\rho)$ is totally bounded iff $\forall i\in \N,\ 
(M_i,\rho_i)$ is totally bounded.

First, we prove the $\Leftarrow$. Let $(M_i,\rho_i)$ totally bounded 
and $X$ any $\varepsilon$-net in $(M,\rho)$. We pick $i_0\in \N$ 
such that $2^{-i_0} < \varepsilon/4$. Now, if $\overline x \neq 
\overline y$, there is $j$ such that $\rho_j(x_j,y_j) \geq 
\varepsilon / (2i_0)$ (if not, then $\sum_0^{i_0} \rho_j(x_j,y_j)/
2^{j} < i_0 \cdot \varepsilon/(2i_0) = \varepsilon/2$ and 
$\sum_{i_0 +1} \rho_j(x_j,y_j) / 2^j \leq \sum 1/2^j \leq 
\varepsilon/4$, therefore $\rho(\overline x, \overline y) < 
\varepsilon$). For all $i\in i_0$ we pick $\varepsilon/(8i_0)$-net
% inclusively maximal
$X_i$ in $M_i$, that is inclusively maximal. $\forall \overline x 
\in X$ there is $f(\overline x)\in \prod X_i$, such that $\forall j\ 
\rho(x_j, f(x)_j) \leq \varepsilon/(8i_0)$. 
If $\overline x\neq \overline y$, then $f(\overline x) \neq f(\overline 
y)$, because 
there is $j,\ \rho(x_j,y_j)\geq \varepsilon /(2i_0)$ and therefore 
$\rho(f(x)_j,f(y)_j) \geq \varepsilon /(4i_0)$, which means
$f$ is injective mapping into finite set, proving that $X$ is 
finite.

To prove $\imply$, we just assume that there is $i_0$ such that 
$(M_{i_0},\rho_{i_0})$ is not totally bounded. Then, there is 
$\varepsilon$-net in $M_{i_0}$ that is infinite. Such net can 
be extended to $M$ by picking arbitrary point in any $M_j$ 
(because every $M_j$ is nonempty), and that is an infinite 
$\varepsilon/2^{i_0}$-net in $M$. 
\qed

\smallskip
$(M,\rho)$ is totally bounded iff $\forall \varepsilon >0$ 
there is a finite cover of $M$ with open $\varepsilon$-balls.
If there is a cover with open balls, then any net cannot contain more 
than one point from any ball and cannot have greater cardinality 
than the cover. On the other hand, if we put an $\varepsilon$-ball 
around every point of some inclusively maximal $\varepsilon$-net, 
we get cover of the same cardinality.
% inclusively maximal

\smallskip
For totally bounded $(M,\rho)$ and $f:\ (M,\rho)\to (\R,|x-y|)$ uniformly 
continuous, $f[M]$ is bounded subset of $\R$. If it were unbounded, 
we could assume that $\forall n\in \N\ \exists x_n\in X\ f(x_n)>n$. 
But then, $\forall n,m\in \N\ |f(x_n)-f(x_m)|\geq 1$. Since $f$ is 
uniformly convergent, $\rho(x,y)< \delta \imply |f(x)-f(y)|<1/2$; and 
for finite cover of $M$ with $\delta/2$ balls we get that one ball 
cannot contain more than one $x_n$.
\qed

\medskip
% nekonecna varle hra

%uplne metr spc
\tpc{Complete metric spaces} We call a sequence $\{x_n|x_n \in M, n\in \N\}$ 
a {\it Cauchy} sequence if $\forall \varepsilon>0\ \exists n_0\ \forall n,m\geq n_0
\ \rho(x_n,x_m) < \varepsilon$. Metric space is {\it complete}, if every Cauchy 
sequence is convergent. 

In any metric space, a convergent sequence is automatically a Cauchy sequence. 
We can take $n_0$ such that $\rho(x,x_n) < \varepsilon/2$, and $\rho(x_n,x_m) 
\leq \rho(x_n,x) + \rho(x_m,x) < \varepsilon$.


%jakysi shit o tom ze R je uplne?
