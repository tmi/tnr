\chapter{Lecture 7}
\centerline{\tt couple of theorems are missing...}
\medskip

c $\imply$ d: for a sequence $x_n$ either there is such $r\in M$ 
so that $x_n = r$ for infinitely many $n$'s or every value is present 
only finitely-times. In the first case, we can simply pick all $x_n$ 
that equal $r$ to get convergent subsequence; in the second case, we 
construct set $X$ as a range of $x_n$. $X$ is therefore infinite and 
has a limit point $x$, i.e. for every $m$ we select $x_m$ so that 
$\rho(x_m,x) < 1/m$ and get converget subsequence.
\smallskip
d $\imply$ e: let $f$ be unbounded, i.e. $\forall n\ \exists x_n\ 
|f(x_n)|>n$, and such a sequence $x_n$ cannot have convergent subsequence, 
since every subsequence is unbounded.
\smallskip
a $\imply$ g: trivial, since $g$ is special case of $a$.
\smallskip
e $\imply$ c: let $X$ be a set without limit point, i.e. $\forall x\ 
\exists \varepsilon\ |B(x,\varepsilon)\cap X| < \infty$. Specially, 
we can say that $X=\{x_i;\, i\in \alpha\},\ \forall i\ \exists \varepsilon_i\ B(x_i,
\varepsilon_i)\cap X = \{x_i\}$. If we set $\delta_i = \min\{\varepsilon_i/3,
1/n\}$, then $B(x_i,\delta_i)$ are pairwise disjoint and $\forall x\in M
\ \exists \varepsilon(x)\ B(x,\varepsilon)$ intersects only finitely many 
$B(x_i,\delta_i)$. Now, we define $f_n(x) = \rho(x,M\bs B(x_n,\delta_n))$ --
this $f$ is continous, as it is defined from metrics; for $x\not\in B(x_n,\delta_n)$ 
equals 0, for $x_n$ it is $\geq \delta_n$. Then, $F(x)=\sum n\cdot 1/\delta_n \cdot 
f_n(x)$ is continous as a sum of continous functions (note that $\forall x\ 
F(x)$ is determined only by finitely many $f_n(x)$ since $f_n(x)$ is non-zero 
only for $x\in B(x_i,\delta_i)$); and therefore has to be bounded. But as 
$F(x_n) \geq n$, we get that $\alpha$ was finite.
\smallskip
c $\imply$ a: let $\cal C$ be an open cover of $M$. First, we prove an 
auxilliary lemma:
$$(\exists \varepsilon > 0)(\forall x\in M)(\exists C\in {\cal C})(B(x,
	\varepsilon)\subseteq C) \leqno (*)$$
{\it Proof:} by contradiction, for $\varepsilon_i=1/i$ we get 
points $x_i$ such that $B(x_i,\varepsilon_i)$ is not subset of any set
in cover. But then $X=\{x_i\}$ is an infinite set without limit point 
-- since limit point $l$ would have to be in some $C$, we could pick 
$\varepsilon$ so that $B(l,\varepsilon)\subseteq C$, $B(l,\varepsilon/2)
\cap X$ is still infinite, and for all $x_i$ in such intersections is 
$B(x_i,\varepsilon/2)\subseteq C$, contradicting that $B(x_i,1/i)$ is 
not subset of any cover for $1/i < \varepsilon/2$.

Now, we pick $\varepsilon$ from $(*)$ and arbitrary $x_1$, for which there
is $C_1 \in \cal C$ so that $B(x_1,\varepsilon)\subseteq C_1$. Now, having 
constructed $C_{1..i-1}$ we proceed constructing $C_i$ -- if $M\neq \bigcup
C_{1..i-1}$, we pick $x_{i+1} \in M\bs \bigcup C_{1..i-1}$, for which there 
is $C_i \in \cal C$ so that $B(x_i,\varepsilon)\subseteq C_i$, this $C_i$ 
is clearly $\neq C_{1..i-1}$. However, if such sequence of $x_i$ is infite, 
we have an infinite $\varepsilon$-net (since every $x_i$ is not in 
$B(x_{1..i-1},\varepsilon)$), which contradicts something ({\tt what?}). 
Therefore, sequence is finite, and $C_{1..i}$ form a finite subcover of $M$.

\smallskip

g $\imply$ {\tt to be finished}

\qed
\medskip

{\bf Theorem.} Metric space is compact iff it is complete and totally bounded.

{\it Proof:} Any compact space has to be totally bounded, since we can pick
cover $\{B(x,\varepsilon);x\in M\}$ which has a finite subcover $S$. A
$\varepsilon$-net cannot have more than one point in each of balls in $S$,
therefore has to be finite. Any compact space has to be complete, since from
Cauchy $x_n$ we can choose a subsequence $x_{n_i}$ with limit $l$; and there is
$n_0$, such that $\forall n_i > n_0\ \rho(x_{n_i},l) < \varepsilon/2$ and
$\forall m,n > n_0\ \rho(x_n,x_m) < \varepsilon/2$. Then is $\forall n>n_0 \
\rho(x_n,l) \leq \rho(x_n,x_{n_i}) + \rho(x_{n_i},l) < \varepsilon$ for some
$n_i > n_0$.

Now for the other implication, we show that in totally bounded space we can
always choose a Cauchy subsequence, which is convergent due to completness. Let
$P$ denote the range of $x_n$, if $P$ is finite, we are done. Otherwise, let 
$M(1/n)$ denote a finite cover with $1/n$-balls (some inclusion-maximall
$1/n$-net). We choose $m_1\in M(1)$ such that $P_1 = P\cap B_1(m_1)$ is infinite
(there is such, since $P$ is infinite and $M(1)$ is cover), and $k_1$ such that
$x_{k_1} \in P_1$. Inductively, we construct infinite $P_i = P_{i-1} \cap
B_{1/i}(m_i \in M(1/i))$ and $x_{k_i} \in P_i$. Then, $x_{k_i}$ is Cauchy, since
for given $\varepsilon > 1/2n$ we have $\forall k_i, k_j > k_n\ x_{k_i},x_{k_j}
\in B_{1/n}(m_n)$.

\qed

\proclaim Corollary. Closed subset of compact space is compact; product of
compact spaces is compact.

\proclaim Theorem (Heine--Borel--Lebesgue). Closed bounded subset of $\R^n$ is
compact.

{\it Proof:} For arbitrary $K$ closed and bounded, there are $a,b$ such that
$a<b$ and $K\subseteq [a,b]^n$. But $[a,b]^n$ is complete and totally bounded,
therefore it is compact, therefore its closed subset is compact.
\qed

{\bf Theorem.} For metric space $(N,\sigma)$, compact metric space
$(M,\rho)$ and $f\,:M\to N$ which is continuous and onto is $(N,\sigma)$
compact. 

{\it Proof:}
Let $G_i$ be open cover of $N$, because of continuity of $f$ we have that
$H = \{f^{-1}[G_i]\}$ is a set of open sets, and since $f$ is onto is $H$ an
open cover of $M$. Therefore we have $H'$ a finite subcover, and its image is a
finite subset of $G$ and still an open cover of $N$.
\qed

{\bf Theorem.} Let $(M,\rho)$ be a compact space, $f$ continous; then $f$
is uniformly continous.

{\it Proof:} $(\forall \varepsilon>0)(\forall x)(\exists \delta_x)$ such that 
$f[B(x,\delta_x)] \subseteq B(f(x),\varepsilon/2)$. Clearly, $B(x,\delta_x)$ 
form an open cover,
therefore there is its finite subcover $S$. Take $\delta =
\min\{\delta_x/2;B(x,\delta_x)\in S\}$, and watch: $\rho(x,y) < \delta
\imply x,y \in B(z,\delta_z),\ B(z,\delta_z)\in S \imply \sigma(f(x),f(y))
\leq \sigma(f(x),f(z)) + \sigma(f(y),f(z)) \leq \varepsilon$.
\qed
